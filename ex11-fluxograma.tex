%%%%%%%%%%%%%%%%%%%%%%%%%%%%%%%%%%%%%%%%%%%%%%%%%%%%%%%%%%%%%%%%%%%%%%%%%%%%%%%%%%%%%%%%
% Criação de Fluxograma usando LaTeX
%
% Assunto: fluxograma sobre o codigo do jogo Battle of Forro
%
%
% Autores:
%     Rafael Brayner Rodrigues
%     Igor Guimaraes Monteiro
%     Anthony Lucas Dos Santos Silva
%
% Coordenação:
%     Prof. Dr. Ruben Carlo Benante
%
% Data: 2024-10-20
%%%%%%%%%%%%%%%%%%%%%%%%%%%%%%%%%%%%%%%%%%%%%%%%%%%%%%%%%%%%%%%%%%%%%%%%%%%%%%%%%%%%%%%%


%%%%%%%%%%%%%%%%%%%%%%%%%%%%%%%%%%%%%%%%%%%%%%%%%%%%%%%%%%%%%%%%%%%%%%%%%%%%%%%%%%%%%%%%
% Para gerar o PDF use o comando make com o makefile configurado:
%
%    $ make ext-programa2-benante-sobrenome1-sobrenome2.pdf
%
% O conteúdo do makefile é composto dos 3 seguintes comandos que ficam assim automatizados:
%    $ pdflatex exN-fluxograma.tex -o exN-fluxograma.pdf
%    $ bibtex biblio
%    $ pdflatex exN-fluxograma.tex -o exN-fluxograma.pdf


%%%%%%%%%%%%%%%%%%%%%%%%%%%%%%%%%%%%%%%%%%%%%%%%%%%%%%%%%%%%%%%%%%%%%%%%%%%%%%%%%%%%%%%%
% preambulo %%%%%%%%%%%%%%%%%%%%%%%%%%%%%%%%%%%%%%%%%%%%%%%%%%%%%%%%%%%%%%%%%%%%%%%%%%%%
\documentclass[a4paper,12pt]{article} %twocolumn
\usepackage[left=2.5cm,right=2cm,top=2.5cm,bottom=2cm]{geometry}
\usepackage[utf8]{inputenc} % letras acentuadas
\usepackage[portuguese]{babel} % tradução de títulos
\usepackage[colorlinks]{hyperref}
\usepackage{tikz} % para adicionar fluxogramas
\usepackage{algorithm} % ambiente para índice de algoritmos
\usepackage{algpseudocode} % fonte e estilo do algoritmo
\usepackage{graphicx} % permite adicionar imagens
\usepackage{indentfirst} % indenta o primeiro parágrafo também
\usepackage{url} % permite adicionar links de URLs e emails
% \usepackage{natbib}
%[noend]

\DeclareUrlCommand\email{\urlstyle{mm}} % comando para email bonito
\floatname{algorithm}{Algoritmo} % tradução da palavra algoritimo no ambiente de índice

\usetikzlibrary{shapes.geometric, shapes.symbols,arrows} % ajuste do tikz para incluir formas e setas

%%%%%%%%%%%%%%%%%%%%%%%%%%%%%%%%%%%%%%%%%%%%%%%%%%%%%%%%%%%%%%%%%%%%%%%%%%%%%%%%%%%%%%%%
% capa %%%%%%%%%%%%%%%%%%%%%%%%%%%%%%%%%%%%%%%%%%%%%%%%%%%%%%%%%%%%%%%%%%%%%%%%%%%%%%%%%
\title{Fluxograma: Battle of Forro}
\author{Rafael Brayner Rodrigues \\ Igor Guimaraes Monteiro \\ Anthony Lucas Dos Santos Silva}

\begin{document}

\maketitle

%%%%%%%%%%%%%%%%%%%%%%%%%%%%%%%%%%%%%%%%%%%%%%%%%%%%%%%%%%%%%%%%%%%%%%%%%%%%%%%%%%%%%%%%
% definicao dos blocos do fluxograma (tikz) %%%%%%%%%%%%%%%%%%%%%%%%%%%%%%%%%%%%%%%%%%%%

\tikzstyle{line} = [draw, -latex']
\tikzstyle{startend} = [draw, ellipse,fill=red!20, minimum height=2em, node distance=1.55cm]
\tikzstyle{print} = [tape, fill=blue!20, draw, draw=black, minimum width=3cm, minimum height=1.4cm, text width=4.5em, text centered, tape bend top=none, tape bend height=0.2cm, node distance=1.55cm]
\tikzstyle{input} = [trapezium, trapezium left angle=60, trapezium right angle=90, minimum width=3cm, minimum height=1cm, text centered, draw=black, fill=blue!30, node distance=1.95cm]
\tikzstyle{process} = [rectangle, minimum width=3cm, minimum height=1cm, text centered, draw=black, fill=orange!30, node distance=1.55cm]

\tikzstyle{block} = [rectangle, draw, fill=blue!20, text width=5em, text centered, rounded corners, minimum height=4em, node distance=1.55cm]
\tikzstyle{decisionb} = [diamond, draw, fill=blue!20, text width=4.5em, text badly centered, inner sep=0pt, node distance=1.55cm]
\tikzstyle{decision} = [diamond, minimum width=3cm, minimum height=1cm, text centered, draw=black, fill=green!30, node distance=2.25cm]
\tikzstyle{empty} = [circle, fill=white, minimum width=0.01mm, node distance=2.55cm]

%%%%%%%%%%%%%%%%%%%%%%%%%%%%%%%%%%%%%%%%%%%%%%%%%%%%%%%%%%%%%%%%%%%%%%%%%%%%%%%%%%%%%%%%
% resumo %%%%%%%%%%%%%%%%%%%%%%%%%%%%%%%%%%%%%%%%%%%%%%%%%%%%%%%%%%%%%%%%%%%%%%%%%%%%%%%

\begin{abstract}

\textbf{Assunto:} programa do jogo Battle of Forro

% descrever em poucas palavras seu projeto aqui

O programa do jogo Battle of Forro mostra primeiro o menu com duas opções, uma de sair e outra de começar, logo após o jogo começar, você sempre terá duas opções de escolha e no final uma opção de sair do jogo. Neste artigo iremos apresentar o seu fluxograma completo.
% e (opcionalmente) o seu algoritmo.

Após a modelagem do fluxograma e desenvolvimento da lógica de programação em algoritmo,
o programa será implementado na Linguagem de Programação \texttt{C}


\textbf{Local:} Escola Politécnica de Pernambuco - UPE/POLI

\textbf{Órgão Financiador:} N/A

\textbf{Caracterização:} Modelagem, Projeto e Implementação de Software em Linguagem \texttt{C}

% Este é o fim do resumo.

\end{abstract}


%%%%%%%%%%%%%%%%%%%%%%%%%%%%%%%%%%%%%%%%%%%%%%%%%%%%%%%%%%%%%%%%%%%%%%%%%%%%%%%%%%%%%%%%
% artigo %%%%%%%%%%%%%%%%%%%%%%%%%%%%%%%%%%%%%%%%%%%%%%%%%%%%%%%%%%%%%%%%%%%%%%%%%%%%%%%
% seção de introdução %%%%%%%%%%%%%%%%%%%%%%%%%%%%%%%%%%%%%%%%%%%%%%%%%%%%%%%%%%%%%%%%%%
\section{Introdução}

% Descrever melhor seu projeto aqui

Este programa é um jogo de texto escrito em C, onde o jogador assume o papel de um centurião enfrentando um poderoso inimigo, o Lawbringer. A função \textbf{exibirMenu()} começa apresentando um menu com duas opções: iniciar a batalha ou sair do jogo. Se o jogador escolhe iniciar a batalha (opção 1), ele é apresentado ao cenário onde seu time foi derrotado e deve enfrentar o Lawbringer, um cavaleiro poderoso com descrições detalhadas de suas habilidades e armamento. O jogador deve escolher entre duas armas: a \texttt{Espada Gladius (opção 1) e a Epsilon Axe (opção 2)}. Se a Gladius for escolhida, o jogador é informado que essa arma é ideal para combates corpo a corpo, permitindo que prossiga para a próxima parte do jogo. Se a Epsilon Axe for escolhida, o jogador não sabe como usá-la e acaba sendo derrotado. Qualquer outra escolha é considerada inválida, levando à morte do jogador.

Após a escolha da arma, o jogador entra em combate e deve decidir como reagir ao ataque do Lawbringer, que está prestes a desferir um golpe poderoso. \texttt{As opções são esquivar-se do ataque, que resulta na morte do jogador, ou atacar com um soco, o que permite derrotar o Lawbringer}. Se o jogador vence, é parabenizado e o jogo oferece a opção de sair. Um loop permite que o jogador insira sua escolha para sair (opção 3), e entradas inválidas resultam em mensagens de erro. Se o jogador escolher sair, uma mensagem de despedida é exibida. A função \textbf{main()} simplesmente chama \textbf{exibirMenu()}, iniciando o jogo. O uso de \textbf{exit(0)} permite que o programa seja encerrado de forma limpa ao final do jogo ou após uma derrota.

%%%%%%%%%%%%%%%%%%%%%%%%%%%%%%%%%%%%%%%%%%%%%%%%%%%%%%%%%%%%%%%%%%%%%%%%%%%%%%%%%%%%%%%%
% seção de objetivos %%%%%%%%%%%%%%%%%%%%%%%%%%%%%%%%%%%%%%%%%%%%%%%%%%%%%%%%%%%%%%%%%%%
\section{Fluxograma}

% adicionar aqui o fluxograma

\begin{tikzpicture}

     % Nodes
    \node (start) [startend, node distance=0.1cm] {início do jogo};
    \node (menu) [process, below of=start, node distance=2cm] {menu};
    \node (opcaoInvalida) [block, left of=menu, xshift=-4cm, node distance=1cm] {opção inválida};
    \node (texto1) [process, below of=menu, node distance=5.5cm] {texto 1};
    \node (texto2) [process, below of=texto1, node distance=5.5cm] {texto 2};
    \node (vitoria) [process, below of=texto2, node distance=5.3cm] {vitória};
    \node (fimJogo) [process, below of=vitoria, node distance=4.5cm] {fim de jogo};
    \node (morte) [startend, right of=texto1, xshift=3cm, yshift=-150, node distance=2cm] {morte};
    \node (decisaoMenu) [decision, below of=menu, yshift=-1cm, node distance=1.9cm] {opção 1 ou 2?};
    \node (decisaoObj) [decision, below of=texto1, yshift=-1cm, node distance=1.7cm] {objeto 1 ou 2?};
    \node (decisaoFinal) [decision, below of=texto2, yshift=-1cm, node distance=1.7cm] {ação 1 ou 2?};
    \node (decisaoSair) [decision, below of=vitoria, yshift=-1cm, node distance=1.1cm] {sair: 3?};

    % setas com numeros
    \path [line] (start) -- (menu);
    \path [line] (menu) -- (decisaoMenu);
    \path [line] (decisaoMenu.west) -- ++(-3.2,0) |- (opcaoInvalida.south) node[near start, left] {maior que 2};
    \path [line] (decisaoMenu.east) -- ++(5,0.01) |- (fimJogo) node[near start, right] {2};
    \path [line] (opcaoInvalida.east) -- ++(1,0) -- (menu.west);
    \path [line] (decisaoMenu.south) -- (texto1) node[midway, right] {1};
    \path [line] (decisaoObj.east) -- ++(1.5,0) |- (morte) node[near start, left] {2};
    \path [line] (texto1) -- (decisaoObj);
    \path [line] (decisaoObj.south) -- (texto2) node[midway, right] {1};
    \path [line] (decisaoFinal.east) -- ++(1.5,0) |- (morte) node[near start, left] {2};
    \path [line] (texto2) -- (decisaoFinal);
    \path [line] (decisaoFinal.south) -- (vitoria) node[midway, right] {1};
    \path [line] (vitoria) -- (decisaoSair);
    \path [line] (decisaoSair.south) -- (fimJogo) node[midway, right] {3};

\end{tikzpicture}



\clearpage % inicia próxima seção em nova página
%%%%%%%%%%%%%%%%%%%%%%%%%%%%%%%%%%%%%%%%%%%%%%%%%%%%%%%%%%%%%%%%%%%%%%%%%%%%%%%%%%%%%%%%
% seção de justificativa %%%%%%%%%%%%%%%%%%%%%%%%%%%%%%%%%%%%%%%%%%%%%%%%%%%%%%%%%%%%%%%
% \section{Algoritmo}

% adicionar aqui o algoritmo (opcional)


% \clearpage % inicia próxima seção em nova página
%%%%%%%%%%%%%%%%%%%%%%%%%%%%%%%%%%%%%%%%%%%%%%%%%%%%%%%%%%%%%%%%%%%%%%%%%%%%%%%%%%%%%%%%
% Autores %%%%%%%%%%%%%%%%%%%%%%%%%%%%%%%%%%%%%%%%%%%%%%%%%%%%%%%%%%%%%%%%%%%%%%%%%%%%%%
\section*{Detalhamento dos Autores}

%%%%%%%%%%%%%%%%%%%%%%%%%%%%%%%%%%%%%%%%%%%%%%%%%%%%%%%%%%%%%%%%%%%%%%%%%%%%%%%%%%%%%%%%
% Discentes %%%%%%%%%%%%%%%%%%%%%%%%%%%%%%%%%%%%%%%%%%%%%%%%%%%%%%%%%%%%%%%%%%%%%%%%%%%%
\subsection*{Discentes}

\begin{enumerate}
    \item \textbf{Nome Completo:} Anthony Lucas Dos Santos Silva
    \begin{description}
        \item [Email:] \email{alss1@poli.br}
       % \item [Currículo Lattes:] \url{http://lattes.cnpq.br/nnnnn}
    \end{description}

    \item \textbf{Nome Completo:} Rafael Brayner Rodrigues
    \begin{description}
        \item [Email:] \email{rbr@poli.br}
       % \item [Currículo Lattes:] \url{http://lattes.cnpq.br/nnnnn}
    \end{description}

    \item \textbf{Nome Completo:} Igor Guimaraes Monteiro
    \begin{description}
        \item [Email:] \email{igm1@poli.br}
       % \item [Currículo Lattes:] \url{http://lattes.cnpq.br/nnnnn}
    \end{description}

    % \item \textbf{Nome Completo:} Fulano de tal
    % \begin{description}
        % \item [Email:] \email{blabla@poli.br}
        % \item [Endereço:]
        % \item [Matrícula:]
        % \item [CPF:]
        % \item [RG:]
        % \item [Telefone:]
        % \item [Currículo Lattes:] \url{http://lattes.cnpq.br/nnnnn}
    % \end{description}

%     \item \textbf{Nome Completo:} Fulano de tal
%     \begin{description}
%         \item [Email:] \email{blabla@poli.br}
%         \item [Endereço:]
%         \item [Matrícula:]
%         \item [CPF:]
%         \item [RG:]
%         \item [Telefone:]
%         \item [Currículo Lattes:] \url{http://lattes.cnpq.br/nnnnn}
%     \end{description}
\end{enumerate}


%%%%%%%%%%%%%%%%%%%%%%%%%%%%%%%%%%%%%%%%%%%%%%%%%%%%%%%%%%%%%%%%%%%%%%%%%%%%%%%%%%%%%%%%
% Docentes %%%%%%%%%%%%%%%%%%%%%%%%%%%%%%%%%%%%%%%%%%%%%%%%%%%%%%%%%%%%%%%%%%%%%%%%%%%%%
\subsection*{Docentes}

\begin{enumerate}
    \item \textbf{Nome Completo:} Ruben Carlo Benante
    \begin{description}
        \item [Email:] \email{rcb@upe.br}
        \item [Matrícula:] 11238-0
        \item [Currículo Lattes:] \url{http://lattes.cnpq.br/3366717378277623}
    \end{description}
\end{enumerate}


%%%%%%%%%%%%%%%%%%%%%%%%%%%%%%%%%%%%%%%%%%%%%%%%%%%%%%%%%%%%%%%%%%%%%%%%%%%%%%%%%%%%%%%%
% referências bibliográficas %%%%%%%%%%%%%%%%%%%%%%%%%%%%%%%%%%%%%%%%%%%%%%%%%%%%%%%%%%%
%\section*{Referências Bibliográficas}

% cite todos, mesmo os não referenciados %%%%%%%%%%%%%%%%%%%%%%%%%%%%%%%%%%%%%%%%%%%%%%%
\nocite{*}


%%%%%%%%%%%%%%%%%%%%%%%%%%%%%%%%%%%%%%%%%%%%%%%%%%%%%%%%%%%%%%%%%%%%%%%%%%%%%%%%%%%%%%%%
% se necessario %%%%%%%%%%%%%%%%%%%%%%%%%%%%%%%%%%%%%%%%%%%%%%%%%%%%%%
% troca autor and autor por autor & autor, na bibliografia. O dcu usa "and"
%\renewcommand{\harvardand}{\&} % troca and pro &. O dcu usa "and"

% Estilos de bibliografia %%%%%%%%%%%%%%%%%%%%%%%%%%%%%%%%%%%%%%%%%%%%%%%%%%%%%%%%%%%%%%
% \bibliographystyle{abnt-alf} % Estilo alfabético da ABNT. Opção [num] para estilo numérico
% \bibliographystyle{apalike}
% \bibliographystyle{dcu} %citacao como (autor and autor, ano). Parece apalike. Rev. Control. Automacao. Use com harvard
% \bibliographystyle{agsm} % padrao harvard fica (autor & autor ano).
\bibliographystyle{acm}

%%%%%%%%%%%%%%%%%%%%%%%%%%%%%%%%%%%%%%%%%%%%%%%%%%%%%%%%%%%%%%%%%%%%%%%%%%%%%%%%%%%%%%%%
% arquivo de banco de dados das referências %%%%%%%%%%%%%%%%%%%%%%%%%%%%%%%%%%%%%%%%%%%%
% renomear para o número do exercício correto
% o arquivo de bibliografia pode se chamar qualquer coisa, isso não muda o comando de gerar o PDF.
% Por exemplo para 'mybiblio.bib', use \bibliography{mybiblio} e os comandos pdflatex e bibtex continuam os mesmos identicos com exN.
\bibliography{biblio}

\end{document}
